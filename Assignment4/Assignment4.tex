\documentclass[journal,12pt,twocolumn]{IEEEtran}

\usepackage{setspace}
\usepackage{gensymb}
\singlespacing
\usepackage[cmex10]{amsmath}

\usepackage{amsthm}

\usepackage{mathrsfs}
\usepackage{txfonts}
\usepackage{stfloats}
\usepackage{bm}
\usepackage{cite}
\usepackage{cases}
\usepackage{subfig}

\usepackage{longtable}
\usepackage{multirow}

\usepackage{enumitem}
\usepackage{mathtools}
\usepackage{steinmetz}
\usepackage{tikz}
\usepackage{circuitikz}
\usepackage{verbatim}
\usepackage{tfrupee}
\usepackage[breaklinks=true]{hyperref}
\usepackage{graphicx}
\usepackage{tkz-euclide}

\usetikzlibrary{calc,math}
\usepackage{listings}
    \usepackage{color}                                            %%
    \usepackage{array}                                            %%
    \usepackage{longtable}                                        %%
    \usepackage{calc}                                             %%
    \usepackage{multirow}                                         %%
    \usepackage{hhline}                                           %%
    \usepackage{ifthen}                                           %%
    \usepackage{lscape}     
\usepackage{multicol}
\usepackage{chngcntr}
\usepackage{hyperref}
\hypersetup{
    colorlinks=true,
    linkcolor=blue,
    filecolor=blue,      
    urlcolor=blue,
}
\DeclareMathOperator*{\Res}{Res}

\renewcommand\thesection{\arabic{section}}
\renewcommand\thesubsection{\thesection.\arabic{subsection}}
\renewcommand\thesubsubsection{\thesubsection.\arabic{subsubsection}}

\renewcommand\thesectiondis{\arabic{section}}
\renewcommand\thesubsectiondis{\thesectiondis.\arabic{subsection}}
\renewcommand\thesubsubsectiondis{\thesubsectiondis.\arabic{subsubsection}}


\hyphenation{op-tical net-works semi-conduc-tor}
\def\inputGnumericTable{}                                 %%

\lstset{
%language=C,
frame=single, 
breaklines=true,
columns=fullflexible
}

\makeatletter
\setlength{\@fptop}{0pt}
\makeatother

\begin{document}


\newtheorem{theorem}{Theorem}[section]
\newtheorem{problem}{Problem}
\newtheorem{proposition}{Proposition}[section]
\newtheorem{lemma}{Lemma}[section]
\newtheorem{corollary}[theorem]{Corollary}
\newtheorem{example}{Example}[section]
\newtheorem{definition}[problem]{Definition}

\newcommand{\BEQA}{\begin{eqnarray}}
\newcommand{\EEQA}{\end{eqnarray}}
\newcommand{\define}{\stackrel{\triangle}{=}}
\bibliographystyle{IEEEtran}
\raggedbottom
\setlength{\parindent}{0pt}
\providecommand{\mbf}{\mathbf}
\providecommand{\pr}[1]{\ensuremath{\Pr\left(#1\right)}}
\providecommand{\qfunc}[1]{\ensuremath{Q\left(#1\right)}}
\providecommand{\sbrak}[1]{\ensuremath{{}\left[#1\right]}}
\providecommand{\lsbrak}[1]{\ensuremath{{}\left[#1\right.}}
\providecommand{\rsbrak}[1]{\ensuremath{{}\left.#1\right]}}
\providecommand{\brak}[1]{\ensuremath{\left(#1\right)}}
\providecommand{\lbrak}[1]{\ensuremath{\left(#1\right.}}
\providecommand{\rbrak}[1]{\ensuremath{\left.#1\right)}}
\providecommand{\cbrak}[1]{\ensuremath{\left\{#1\right\}}}
\providecommand{\lcbrak}[1]{\ensuremath{\left\{#1\right.}}
\providecommand{\rcbrak}[1]{\ensuremath{\left.#1\right\}}}
\theoremstyle{remark}
\newtheorem{rem}{Remark}
\newcommand{\sgn}{\mathop{\mathrm{sgn}}}
\providecommand{\abs}[1]{$\left\vert#1\right\vert$}
\providecommand{\res}[1]{\Res\displaylimits_{#1}} 
\providecommand{\norm}[1]{$\left\lVert#1\right\rVert$}
%\providecommand{\norm}[1]{\lVert#1\rVert}
\providecommand{\mtx}[1]{\mathbf{#1}}
\providecommand{\mean}[1]{$E\left[ #1 \right]$}
\providecommand{\fourier}{\overset{\mathcal{F}}{ \rightleftharpoons}}
%\providecommand{\hilbert}{\overset{\mathcal{H}}{ \rightleftharpoons}}
\providecommand{\system}{\overset{\mathcal{H}}{ \longleftrightarrow}}
	%\newcommand{\solution}[2]{\textbf{Solution:}{#1}}
\newcommand{\solution}{\noindent \textbf{Solution: }}
\newcommand{\cosec}{\,\text{cosec}\,}
\providecommand{\dec}[2]{\ensuremath{\overset{#1}{\underset{#2}{\gtrless}}}}
\newcommand{\myvec}[1]{\ensuremath{\begin{pmatrix}#1\end{pmatrix}}}
\newcommand{\mydet}[1]{\ensuremath{\begin{vmatrix}#1\end{vmatrix}}}
\newcommand*{\permcomb}[4][0mu]{{{}^{#3}\mkern#1#2_{#4}}}
\newcommand*{\perm}[1][-3mu]{\permcomb[#1]{P}}
\newcommand*{\comb}[1][-1mu]{\permcomb[#1]{C}}
\numberwithin{equation}{subsection}
\makeatletter
\@addtoreset{figure}{problem}
\makeatother
\let\StandardTheFigure\thefigure
\let\vec\mathbf
\renewcommand{\thefigure}{\theproblem}
\def\putbox#1#2#3{\makebox[0in][l]{\makebox[#1][l]{}\raisebox{\baselineskip}[0in][0in]{\raisebox{#2}[0in][0in]{#3}}}}
     \def\rightbox#1{\makebox[0in][r]{#1}}
     \def\centbox#1{\makebox[0in]{#1}}
     \def\topbox#1{\raisebox{-\baselineskip}[0in][0in]{#1}}
     \def\midbox#1{\raisebox{-0.5\baselineskip}[0in][0in]{#1}}
\vspace{3cm}
\title{Assignment 4}
\author{Perambuduri Srikaran - AI20BTECH11018}
\maketitle
\newpage
\bigskip
\renewcommand{\thefigure}{\theenumi}
\renewcommand{\thetable}{\theenumi}
%Download all python codes from
%\begin{lstlisting}
%https://github.com/srikaran-p/AI1103/tree/main/Assign%ment4/codes
%\end{lstlisting}
%and latex codes from 
Download all python codes from
\begin{lstlisting}
https://github.com/srikaran-p/AI1103/tree/main/Assignment4/codes
\end{lstlisting}
and latex codes from 
\begin{lstlisting}
https://github.com/srikaran-p/AI1103/tree/main/Assignment4
\end{lstlisting}
\section*{Problem}
(STATS P1 IES ISS 2019 Q16) Let $X$ be a Poisson random variable with p.m.f
\begin{align}
\label{eq:1}
P(X=k) = 
    \begin{cases} 
      \frac{e^{-\lambda}\lambda^{k}}{k!},& k=0,1,2,...;  \lambda > 0\\
      0 & \text{otherwise}
   \end{cases}
\end{align}
If $Y = X^2 + 3$, then what is $P(Y=y)$ equal to?
\begin{enumerate}[label={(\Alph*)}]
    \item $\frac{e^{-\lambda}\lambda^{\sqrt{y-3}}}{\sqrt{\brak{y-3}}!}$, for $y =$ \cbrak{3,4,7,12,...}
    \item $\frac{e^{-\lambda}\lambda^{-\sqrt{y-3}}}{\sqrt{\brak{3-y}}!}$, for $y =$ \cbrak{3,4,7,12,...}
    \item $\frac{e^{-\lambda}\lambda^{\sqrt{3-y}}}{\sqrt{\brak{3-y}}!}$, for $y =$ \cbrak{4,7,12,...}
    \item $\frac{e^{-\lambda}\lambda^{-\sqrt{3-y}}}{\sqrt{\brak{3-y}}!}$, for $y =$ \cbrak{4,7,12,...}
\end{enumerate}
\section*{Solution}
\begin{align}
    Y = X^2 + 3\\
    X = \sqrt{Y - 3}
\end{align}
We can substitute $k = \sqrt{y-3}$ in \eqref{eq:1}
\begin{align}
p_Y(y) = 
    \begin{cases} 
      \frac{e^{-\lambda}\lambda^{\sqrt{y-3}}}{\sqrt{\brak{y-3}}!}, & y=3,4,7,12,...\\
      0&\text{otherwise}
   \end{cases}
\end{align}
Hence, the correct option is \brak{A}.
\begin{figure}[hb]
    \centering
    \includegraphics[width=\columnwidth]{FigureX.png}
    \caption{Poisson stem plot for X \brak{\lambda = 5}}
    \label{fig:plot1}
\end{figure}
\begin{figure}[hb]
    \centering
    \includegraphics[width=\columnwidth]{FigureComp.png}
    \caption{Poisson stem plot for Y (Simulated and Theoretical) \brak{\lambda = 5}}
    \label{fig:plot3}
\end{figure}
\end{document}