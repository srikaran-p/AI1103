\documentclass[journal,12pt,twocolumn]{IEEEtran}

\usepackage{setspace}
\usepackage{gensymb}
\singlespacing
\usepackage[cmex10]{amsmath}

\usepackage{amsthm}

\usepackage{mathrsfs}
\usepackage{txfonts}
\usepackage{stfloats}
\usepackage{bm}
\usepackage{cite}
\usepackage{cases}
\usepackage{subfig}

\usepackage{longtable}
\usepackage{multirow}

\usepackage{enumitem}
\usepackage{mathtools}
\usepackage{steinmetz}
\usepackage{tikz}
\usepackage{circuitikz}
\usepackage{verbatim}
\usepackage{tfrupee}
\usepackage[breaklinks=true]{hyperref}
\usepackage{graphicx}
\usepackage{tkz-euclide}

\usetikzlibrary{calc,math}
\usepackage{listings}
    \usepackage{color}                                            %%
    \usepackage{array}                                            %%
    \usepackage{longtable}                                        %%
    \usepackage{calc}                                             %%
    \usepackage{multirow}                                         %%
    \usepackage{hhline}                                           %%
    \usepackage{ifthen}                                           %%
    \usepackage{lscape}     
\usepackage{multicol}
\usepackage{chngcntr}
\usepackage{hyperref}
\hypersetup{
    colorlinks=true,
    linkcolor=blue,
    filecolor=blue,      
    urlcolor=blue,
}
\DeclareMathOperator*{\Res}{Res}

\renewcommand\thesection{\arabic{section}}
\renewcommand\thesubsection{\thesection.\arabic{subsection}}
\renewcommand\thesubsubsection{\thesubsection.\arabic{subsubsection}}

\renewcommand\thesectiondis{\arabic{section}}
\renewcommand\thesubsectiondis{\thesectiondis.\arabic{subsection}}
\renewcommand\thesubsubsectiondis{\thesubsectiondis.\arabic{subsubsection}}


\hyphenation{op-tical net-works semi-conduc-tor}
\def\inputGnumericTable{}                                 %%

\lstset{
%language=C,
frame=single, 
breaklines=true,
columns=fullflexible
}

\makeatletter
\setlength{\@fptop}{0pt}
\makeatother

\begin{document}


\newtheorem{theorem}{Theorem}[section]
\newtheorem{problem}{Problem}
\newtheorem{proposition}{Proposition}[section]
\newtheorem{lemma}{Lemma}[section]
\newtheorem{corollary}[theorem]{Corollary}
\newtheorem{example}{Example}[section]
\newtheorem{definition}[problem]{Definition}

\newcommand{\BEQA}{\begin{eqnarray}}
\newcommand{\EEQA}{\end{eqnarray}}
\newcommand{\define}{\stackrel{\triangle}{=}}
\bibliographystyle{IEEEtran}
\raggedbottom
\setlength{\parindent}{0pt}
\providecommand{\mbf}{\mathbf}
\providecommand{\pr}[1]{\ensuremath{\Pr\left(#1\right)}}
\providecommand{\qfunc}[1]{\ensuremath{Q\left(#1\right)}}
\providecommand{\sbrak}[1]{\ensuremath{{}\left[#1\right]}}
\providecommand{\lsbrak}[1]{\ensuremath{{}\left[#1\right.}}
\providecommand{\rsbrak}[1]{\ensuremath{{}\left.#1\right]}}
\providecommand{\brak}[1]{\ensuremath{\left(#1\right)}}
\providecommand{\lbrak}[1]{\ensuremath{\left(#1\right.}}
\providecommand{\rbrak}[1]{\ensuremath{\left.#1\right)}}
\providecommand{\cbrak}[1]{\ensuremath{\left\{#1\right\}}}
\providecommand{\lcbrak}[1]{\ensuremath{\left\{#1\right.}}
\providecommand{\rcbrak}[1]{\ensuremath{\left.#1\right\}}}
\theoremstyle{remark}
\newtheorem{rem}{Remark}
\newcommand{\sgn}{\mathop{\mathrm{sgn}}}
\providecommand{\abs}[1]{$\left\vert#1\right\vert$}
\providecommand{\res}[1]{\Res\displaylimits_{#1}} 
\providecommand{\norm}[1]{$\left\lVert#1\right\rVert$}
%\providecommand{\norm}[1]{\lVert#1\rVert}
\providecommand{\mtx}[1]{\mathbf{#1}}
\providecommand{\mean}[1]{$E\left[ #1 \right]$}
\providecommand{\fourier}{\overset{\mathcal{F}}{ \rightleftharpoons}}
%\providecommand{\hilbert}{\overset{\mathcal{H}}{ \rightleftharpoons}}
\providecommand{\system}{\overset{\mathcal{H}}{ \longleftrightarrow}}
	%\newcommand{\solution}[2]{\textbf{Solution:}{#1}}
\newcommand{\solution}{\noindent \textbf{Solution: }}
\newcommand{\cosec}{\,\text{cosec}\,}
\providecommand{\dec}[2]{\ensuremath{\overset{#1}{\underset{#2}{\gtrless}}}}
\newcommand{\myvec}[1]{\ensuremath{\begin{pmatrix}#1\end{pmatrix}}}
\newcommand{\mydet}[1]{\ensuremath{\begin{vmatrix}#1\end{vmatrix}}}
\newcommand*{\permcomb}[4][0mu]{{{}^{#3}\mkern#1#2_{#4}}}
\newcommand*{\perm}[1][-3mu]{\permcomb[#1]{P}}
\newcommand*{\comb}[1][-1mu]{\permcomb[#1]{C}}
\numberwithin{equation}{subsection}
\makeatletter
\@addtoreset{figure}{problem}
\makeatother
\let\StandardTheFigure\thefigure
\let\vec\mathbf
\renewcommand{\thefigure}{\theproblem}
\def\putbox#1#2#3{\makebox[0in][l]{\makebox[#1][l]{}\raisebox{\baselineskip}[0in][0in]{\raisebox{#2}[0in][0in]{#3}}}}
     \def\rightbox#1{\makebox[0in][r]{#1}}
     \def\centbox#1{\makebox[0in]{#1}}
     \def\topbox#1{\raisebox{-\baselineskip}[0in][0in]{#1}}
     \def\midbox#1{\raisebox{-0.5\baselineskip}[0in][0in]{#1}}
\vspace{3cm}
\title{Assignment 4}
\author{Perambuduri Srikaran - AI20BTECH11018}
\maketitle
\newpage
\bigskip
\renewcommand{\thefigure}{\theenumi}
\renewcommand{\thetable}{\theenumi}
%Download all python codes from
%\begin{lstlisting}
%https://github.com/srikaran-p/AI1103/tree/main/Assign%ment4/codes
%\end{lstlisting}
%and latex codes from 
Download latex codes from
\begin{lstlisting}
https://github.com/srikaran-p/AI1103/tree/main/Assignment4
\end{lstlisting}
\section*{Problem}
(GATE-MA 2015 Q17) Let $\tau_1$ be the usual topology on $\mathbb{R}$. Let $\tau_2$ be the topology on $\mathbb{R}$ generated by $\mathcal{B}$ = $\cbrak{\lsbrak{a}, \rbrak{b} \subset \mathbb{R} : -\infty < a < b < \infty}$. Then the set $\cbrak{x \in \mathbb{R} : 4sin^2x \leq 1} \cup \cbrak{\frac{\pi}{2}}$ is
\begin{enumerate}[label={(\Alph*)}]
    \item closed in $\brak{\mathbb{R}, \tau_1}$ but NOT in $\brak{\mathbb{R}, \tau_2}$
    \item closed in $\brak{\mathbb{R}, \tau_2}$ but NOT in $\brak{\mathbb{R}, \tau_1}$
    \item closed in both $\brak{\mathbb{R}, \tau_1}$ and $\brak{\mathbb{R}, \tau_2}$
    \item neither closed in $\brak{\mathbb{R}, \tau_1}$ nor closed in $\brak{\mathbb{R}, \tau_2}$
\end{enumerate}
\section*{Solution}
Let $A$ be the set of all the solutions of the given inequality,
\begin{multline}
    A = \bigcup_{n \in \mathbb{Z}} \sbrak{2n\pi - \frac{\pi}{6}, 2n\pi + \frac{\pi}{6}} +\\ \bigcup_{n \in \mathbb{Z}} \sbrak{2n\pi + \frac{5\pi}{6}, 2n\pi + \frac{7\pi}{6}} + \cbrak{\frac{\pi}{2}}
\end{multline}
\begin{multline}
    A' = \brak{\frac{-5\pi}{6},\frac{-\pi}{6}} + \brak{\frac{\pi}{6}, \frac{\pi}{2}} + \brak{\frac{\pi}{2}, \frac{5\pi}{6}} + \\ \bigcup_{n \in \mathbb{Z} - \cbrak{0}} \brak{2n\pi - \frac{5\pi}{6}, 2n\pi - \frac{\pi}{6}} +\\ \bigcup_{n \in \mathbb{Z} - \cbrak{0}} \brak{2n\pi + \frac{\pi}{6}, 2n\pi + \frac{5\pi}{6}}
\end{multline}
\subsection{Definition}
\label{sec:def1}
A set $U$ of real numbers is said to be open if for all $x \in U$, there exists $\delta(x) > 0$ such that $\brak{x - \delta(x), x + \delta(x)} \subset U$.\\
The intervals in $A'$ are open sets by \ref{sec:def1}.
\subsection{Theorem}
\label{sec:thrm1}
If $\cbrak{U_{\alpha}}$ is any collection (finite, infinite, countable or uncountable) of open sets, then $\bigcup_{\alpha}U_{\alpha}$ is an open set.\\
$A'$ is an open set by \ref{sec:thrm1}.
\begin{align}
    A' \in \tau_1
\end{align}
A' is not closed in $\brak{\mathbb{R}, \tau_1}$.\\
$\implies$ A is closed in $\brak{\mathbb{R}, \tau_1}$.
\begin{align}
    \brak{a,b} = \bigcup_{n = 1}^{\infty}\left[a + \frac{1}{n}, b\right)\label{eq:1}
\end{align}
The intervals in $A'$ can be written as \eqref{eq:1}.
\begin{align}
    A' \in \tau_2
\end{align}
$A'$ is not closed in $\brak{\mathbb{R}, \tau_2}$.\\
$\implies$ A is closed in $\brak{\mathbb{R}, \tau_2}$.\\
Hence, option $\brak{C}$ is correct.
\end{document}