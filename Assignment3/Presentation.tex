\documentclass{beamer}
\usepackage{listings}
\lstset{
%language=C,
frame=single, 
breaklines=true,
columns=fullflexible
}
\usepackage{subcaption}
\usepackage{url}
\usepackage{tikz}
\usepackage{tkz-euclide} % loads  TikZ and tkz-base
%\usetkzobj{all}
\usetikzlibrary{calc,math}
\usepackage{float}
\newcommand\norm[1]{\left\lVert#1\right\rVert}
\renewcommand{\vec}[1]{\mathbf{#1}}
\usepackage[export]{adjustbox}
\usepackage[utf8]{inputenc}
\usepackage{amsmath}
\usetheme{Boadilla}

\title{GATE MA 2015 Q10}
\author{Perambuduri Srikaran}
\institute{IITH AI}
\date{April 6, 2021}
\begin{document}
\providecommand{\pr}[1]{\ensuremath{\Pr\left(#1\right)}}
\providecommand{\qfunc}[1]{\ensuremath{Q\left(#1\right)}}
\providecommand{\sbrak}[1]{\ensuremath{{}\left[#1\right]}}
\providecommand{\lsbrak}[1]{\ensuremath{{}\left[#1\right.}}
\providecommand{\rsbrak}[1]{\ensuremath{{}\left.#1\right]}}
\providecommand{\brak}[1]{\ensuremath{\left(#1\right)}}
\providecommand{\lbrak}[1]{\ensuremath{\left(#1\right.}}
\providecommand{\rbrak}[1]{\ensuremath{\left.#1\right)}}
\providecommand{\cbrak}[1]{\ensuremath{\left\{#1\right\}}}
\providecommand{\lcbrak}[1]{\ensuremath{\left\{#1\right.}}
\providecommand{\rcbrak}[1]{\ensuremath{\left.#1\right\}}}
\begin{frame}
\titlepage
\end{frame}
\section{Question}
\begin{frame}
\frametitle{Question}
\begin{block}{}
Let $X$ be a random variable having the distribution function
\begin{align}
F(x) = 
    \begin{cases} 
      0 & \text{if }x < 0 \\
      \frac{1}{4} & \text{if } 0\leq x < 1 \\
      \frac{1}{3} & \text{if } 1\leq x < 2 \\
      \frac{1}{2} & \text{if } 2\leq x < \frac{11}{3} \\
      1 & \text{if } x \geq \frac{11}{3} \\
   \end{cases}
\end{align}
Then $E(X)$ is equal to ...
\end{block}
\end{frame}
\subsection*{Solution1}
\begin{frame}[fragile]
\frametitle{Solution 1}
\begin{flushleft}
We will use dirac $\delta$ functions to solve the question.\\
Consider a unit step function $u(x)$,
\begin{align}
u(x) = 
    \begin{cases}
        1 & x \geq 0 \\
        0 & \text{otherwise}
    \end{cases}
\end{align}
We will remove the discontinuity at $x=0$ by defining $u_{\alpha}(x)$ for any $\alpha > 0$,
\begin{align}
u_{\alpha}(x) &= 
    \begin{cases}
        1 & x > \frac{\alpha}{2} \\
        \frac{1}{\alpha}(x + \frac{\alpha}{2}) & -\frac{\alpha}{2} \leq x \leq \frac{\alpha}{2} \\
        0 & x < -\frac{\alpha}{2}
    \end{cases}
\end{align}
\end{flushleft}
\end{frame}
\begin{frame}[fragile]
\frametitle{Solution 1 contd...}
\begin{flushleft}
\begin{align}
\delta_{\alpha}(x) &= \frac{du_{\alpha}(x)}{dx}\\
\delta_{\alpha}(x) &=
    \begin{cases}
        \frac{1}{\alpha} & -\frac{\alpha}{2} \leq x \leq \frac{\alpha}{2} \\
        0 & |x| > \frac{\alpha}{2}
    \end{cases}\\
u(x) &= \lim_{\alpha \to 0} u_{\alpha}(x)\\
\delta (x) &= \lim_{\alpha \to 0} \delta_{\alpha}(x)
\end{align}
\end{flushleft}
\end{frame}
\begin{frame}[fragile]
\frametitle{Lemma}
\begin{align}
    \int_{-\infty}^{\infty}g(x)\delta(x-x_0)dx = g(x_0) \label{eq:1}
\end{align}
Proof:\newline
Let $I$ be the value of the above integral.
\begin{align}
I &= \lim_{\alpha \to 0}\sbrak{\int_{-\infty}^{\infty}g(x)\delta_{\alpha}(x-x_0)dx}\\
&= \lim_{\alpha \to 0}\sbrak{\int_{x_0 - \frac{\alpha}{2}}^{x_0+\frac{\alpha}{2}}\frac{g(x)}{\alpha}dx}\\
\int_{x_0 - \frac{\alpha}{2}}^{x_0+\frac{\alpha}{2}}\frac{g(x)}{\alpha}dx &= \alpha\frac{g(x_{\alpha})}{\alpha} = g(x_\alpha)
\end{align}
Here, $x_{\alpha} \in \brak{x_0 - \frac{\alpha}{2}, x_0 + \frac{\alpha}{2}}$
\begin{align}
I = \lim_{\alpha \to 0}g(x_{\alpha}) = g(x_0)
\end{align}
\end{frame}
\begin{frame}[fragile]
\frametitle{Solution 1 contd...}
\begin{align}
f_X(x) &= \frac{dF(x)}{dx}\\
f_{X}(x) &= \frac{1}{4}\delta(x) + \frac{1}{12}\delta(x - 1) + \frac{1}{6}\delta(x-2)+\frac{1}{2}\delta\brak{x-\frac{11}{3}} + 0
\end{align}
We will use \eqref{eq:1} to compute $E(X)$
\begin{align}
    E(X) &= \int_{-\infty}^{\infty}xf_X(x)dx\\
    E(X) &= \int_{-\infty}^{\infty}\frac{1}{4}x\delta(x) + \frac{1}{12}x\delta(x - 1) + \frac{1}{6}x\delta(x-2) + \frac{1}{2}x\delta\brak{x-\frac{11}{3}}dx\\
    E(X) &= {\frac{1}{4}}\times 0 + {\frac{1}{12}}\times 1 + {\frac{1}{6}}\times 2 + {\frac{1}{2}}\times{\frac{11}{3}}\\
    E(X) &= 2.25
\end{align}
\end{frame}
\subsection*{Solution2}
\begin{frame}[fragile]
\frametitle{Solution 2}
\begin{flushleft}
We will use Riemann Stieltjes integral.\newline
Suppose $F(x)$ is continuously differentiable on \sbrak{a,b} except at a finite number of points $c_1, c_2, \dots c_n$.
\begin{multline}
    \int_{a}^{b}g(x)dF(x) = \int_{a}^{c_1}g(x)dF(x) + g(c_1)\sbrak{F(c_1) - \lim_{\substack{x \to c_1 \\ x < c_1}}F(x)} + \\ \int_{c_1}^{c_2}g(x)dF(x) + g(c_2)\sbrak{F(c_2) - \lim_{\substack{x \to c_2 \\ x < c_2}}F(x)} + \dots +\\ \int_{c_{n-1}}^{c_n}g(x)dF(x) + g(c_n)\sbrak{F(c_n) - \lim_{\substack{x \to c_n \\ x < c_n}}F(x)} +  \int_{c_n}^{b}g(x)dF(x)
\end{multline}
\end{flushleft}
\end{frame}
\begin{frame}[fragile]
\frametitle{Solution 2 contd...}
\newline
\begin{align}
    E(g(X)) &= \int_{-\infty}^{\infty} g(x)dF(x)\\
    E(X) &= \int_{-\infty}^{\infty} xdF(x)
\end{align}
\end{frame}
\begin{frame}[fragile]
\frametitle{Solution 2 contd...}
\begin{flushleft}
\begin{multline}
    E(X) = \int_{-\infty}^{0}xdF(x) + 0\sbrak{F(0) - \lim_{\substack{x \to 0 \\ x < 0}}F(x)} + \\ \int_{0}^{1}xdF(x) + 1\sbrak{F(1) - \lim_{\substack{x \to 1 \\ x < 1}}F(x)} + \\ \int_{1}^{2}xdF(x) + 2\sbrak{F(2) - \lim_{\substack{x \to 2 \\ x < 2}}F(x)} + \\ \int_{2}^{\frac{11}{3}}xdF(x) +  \frac{11}{3}\sbrak{F\brak{\frac{11}{3}} - \lim_{\substack{x \to \frac{11}{3} \\ x < \frac{11}{3}}}F(x)}+\\\int_{\frac{11}{3}}^{\infty}xdF(x)
\end{multline}
\end{flushleft}
\end{frame}
\begin{frame}[fragile]
\frametitle{Solution 2 contd...}
\begin{flushleft}
\begin{multline}
    \int_{-\infty}^{0}xdF(x) = \int_{0}^{1}xdF(x) = \int_{1}^{2}xdF(x) =  \int_{2}^{\frac{11}{3}}xdF(x) =\\ \int_{\frac{11}{3}}^{\infty}xdF(x) = 0
\end{multline}
\begin{align}
    E(X) = 2.25
\end{align}
\end{flushleft}
\end{frame}
\end{document}